\PassOptionsToPackage{unicode=true}{hyperref} % options for packages loaded elsewhere
\PassOptionsToPackage{hyphens}{url}
%
\documentclass[a4paper]{article}
\usepackage{lmodern}
\usepackage{amssymb,amsmath}
\usepackage{ifxetex,ifluatex}
\usepackage{fixltx2e} % provides \textsubscript
\ifnum 0\ifxetex 1\fi\ifluatex 1\fi=0 % if pdftex
  \usepackage[T1]{fontenc}
  \usepackage[utf8]{inputenc}
  \usepackage{textcomp} % provides euro and other symbols
\else % if luatex or xelatex
  \usepackage{unicode-math}
  \defaultfontfeatures{Ligatures=TeX,Scale=MatchLowercase}
\fi
% use upquote if available, for straight quotes in verbatim environments
\IfFileExists{upquote.sty}{\usepackage{upquote}}{}
% use microtype if available
\IfFileExists{microtype.sty}{%
\usepackage[]{microtype}
\UseMicrotypeSet[protrusion]{basicmath} % disable protrusion for tt fonts
}{}
\IfFileExists{parskip.sty}{%
\usepackage{parskip}
}{% else
\setlength{\parindent}{0pt}
\setlength{\parskip}{6pt plus 2pt minus 1pt}
}
\usepackage{hyperref}
\hypersetup{
            pdftitle={Plant traits, vegetation, and C flux data from warming experiments and elevation gradients in Adventdalen, Svalbard.},
            pdfborder={0 0 0},
            breaklinks=true}
\urlstyle{same}  % don't use monospace font for urls
\usepackage[margin=0.8in]{geometry}
\usepackage{longtable,booktabs}
% Fix footnotes in tables (requires footnote package)
\IfFileExists{footnote.sty}{\usepackage{footnote}\makesavenoteenv{longtable}}{}
\usepackage{graphicx,grffile}
\makeatletter
\def\maxwidth{\ifdim\Gin@nat@width>\linewidth\linewidth\else\Gin@nat@width\fi}
\def\maxheight{\ifdim\Gin@nat@height>\textheight\textheight\else\Gin@nat@height\fi}
\makeatother
% Scale images if necessary, so that they will not overflow the page
% margins by default, and it is still possible to overwrite the defaults
% using explicit options in \includegraphics[width, height, ...]{}
\setkeys{Gin}{width=\maxwidth,height=\maxheight,keepaspectratio}
\setlength{\emergencystretch}{3em}  % prevent overfull lines
\providecommand{\tightlist}{%
  \setlength{\itemsep}{0pt}\setlength{\parskip}{0pt}}
\setcounter{secnumdepth}{5}
% Redefines (sub)paragraphs to behave more like sections
\ifx\paragraph\undefined\else
\let\oldparagraph\paragraph
\renewcommand{\paragraph}[1]{\oldparagraph{#1}\mbox{}}
\fi
\ifx\subparagraph\undefined\else
\let\oldsubparagraph\subparagraph
\renewcommand{\subparagraph}[1]{\oldsubparagraph{#1}\mbox{}}
\fi

% set default figure placement to htbp
\makeatletter
\def\fps@figure{htbp}
\makeatother

\usepackage{setspace}\doublespacing
\usepackage{lineno}
\linenumbers

\title{Plant traits, vegetation, and C flux data from warming experiments and elevation gradients in Adventdalen, Svalbard.}
\author{Halbritter, A. H.\footnote{Corresponding author: \href{mailto:aud.halbritter@uib.no}{\nolinkurl{aud.halbritter@uib.no}}} \(^1\), Althuizen, I. \(^2\), Pernille Bronken Eidesen, Geir Wing Gabrielsen, Jonathan Henn, Ingibjörg Svala Jónsdóttir, Kari Klanderud, Marc Macias-Fauria, Yadvinder Malhi, Brian Salvin Maitner, Sean Michaletz, Lori Patrick, Ruben Roos, Richard Telford, Polly Bass, Katrín Björnsdóttir, Lucely Lucero Vilca Bustamante, Adam Chmurzynski, Shuli Chen, Siri Vatsø Haugum, Julia Kemppinen, Kai Lepley, Yaoqi Li, Mary Linabury, Ilaíne Silveira Matos, Barbara M. Neto-Bradley, Molly Ng, Pekka Niittynen, Silje Östman, Karolína Pánková, Nina Roth, Matiss Castorena Salaks, Marcus Spiegel, Eleanor Thomson, Alexander Sæle Vågenes, Brian Enquist, Vigdis Vandvik}
\date{23 September 2020}

\begin{document}
\maketitle
\begin{abstract}
to be added
\end{abstract}

\textbf{Keywords:} biodiviersity, ecosystem function, arctic, traits, data

\hypertarget{background-and-summary}{%
\subsection{Background and Summary}\label{background-and-summary}}

Global climate warming is predicted to be more amplified in the Arctic compared to other regions of the world (IPCC, 2013) and has wide-ranging impacts on biodiversity and ecosystem functioning of arctic ecosystems.
Warmer climate is affecting phenology1,2, species ranges3,4, local plant abundance and biodiversity5,6, and ecosystem carbon, nutrient, and water fluxes7.
The current pace of climate change in the Arctic necessitates proper assessments of arctic plant and ecosystem responses to a warmer climate.
However, there is substantial variation between systems and regions in magnitude and directions of the response to a given climate change8--10.
These context-dependencies limit our understanding of observed climate change impacts and our ability to forecast future trends in biodiversity and ecosystem functioning11,12.
Comparisons across sites and using different approaches, are critically important to understand what drives the variation in biodiversity and ecosystem response to climate change.

\hypertarget{this-para-should-maybe-be-more-on-the-different-data-sets-that-we-are-collecting-community-traits-remote-sensing}{%
\section{this para should maybe be more on the different data sets that we are collecting: community, traits, remote sensing\ldots{}}\label{this-para-should-maybe-be-more-on-the-different-data-sets-that-we-are-collecting-community-traits-remote-sensing}}

Trait-based approaches offer opportunities for generalization and improved process-based understanding in global change ecology; and in particular for understanding drivers and consequences of context-dependencies in climate responses.
Functional traits underlie variation among individuals in their ability to survive, reproduce, and function under different environmental conditions13--15.
Because traits vary both within and between species, and can be affected by environment, evolutionary history and plasticity16,17, trait-based approaches offer great opportunity for improved understanding of the drivers, constraints and consequences of variability in biodiversity and ecosystem responses to climate change.
In arctic regions, where temperature is a major limiting factor, traits associated with the leaf economics spectrum, a set of leaf traits that characterize individuals along a continuum from `fast' to `slow' photosynthetic and tissue turnover rates and life histories18--20, should be particularly relevant for responses to climatic warming.
Functional traits thus improve our mechanistic understanding of species' response to and functioning under climate change by providing a linkage between the phenotypes of individuals and the environment.
An advantage of a trait-based approach is that trait measures offer a `common currency' that facilitates comparison across regions and systems that may not share many taxa, as well as across experimental approaches.

Seabirds are important providers of nutrients to terrestrial vegetation, especially in the High Arctic.
Guano deposition provides ocean-derived nutrients to ecosystems, but this effect is not homogeneous.
Many seabirds nest in cliffs along oceans, creating a gradient of nutrients leading away from the colony; however, the specific influence of seabird nutrient deposition is still debated.
Research conflicts regarding seabird influence on vegetation. More research is needed to clarify seabird influence in Arctic ecosystems.

In this paper, we report data on biodiversity and ecosystem function from one warming experiment in Endalen and two elevational gradients in Adventdalen and Bjørndalen (?), Longyearbyen Svalbard.
For the warming experiment, we an already existing International Tundra Experiment (ITEX) site with passive Open Top Chambers (OTC) warming plots (+ 1-2 °C relative to control plots) located in three different habitats (snowbed and Dryas and Cassiope heath).
The ITEX site in Endalen has been running since 2002 and is one of few High Arctic sites included in the ITEX network.
For the two elevational gradients, seven or five site were set up between sea level and xxx and yy m a.s.l, respectively.
The gradient in Bjørndalen was set up below a birdcliff, representing a nutrient gradient, which allows studying the effect of seabirds on plant communities and ecosystem functioning.
In both experimental warming approaches, plant species richness, cover, plant functional traits, ecosystem carbon fluxes, remote sensing and spectral reflectance and associated climate data were measured in 2018. At the ITEX site, plant community composition and climate data had been recorded since 2003.
The community composition and ecosystem flux data are comparable with data from previous courses in Gongga Mountains, China (2015, 2016) and Wayquecha, Peru (PFTC3 and 5) in 2018 and 2020 as well as with upcoming courses.

\hypertarget{methods}{%
\subsection{Methods}\label{methods}}

\hypertarget{research-site-and-experimental-design}{%
\subsubsection{Research site and experimental design}\label{research-site-and-experimental-design}}

This study was conducted at three high-arctic sites near Longyearbyen, Svalbard.
One site was located in Endalen at an already existing ITEX site (Björnsdóttir 2018) comparing artificially heated vegetation and control vegetation in three different habitats (Dryas heath, Cassiope heath and snowbed).
Two of the sites were located along elevational gradients, Lindholmhøgda and Bjørndalen.
The Bjørndalen gradient was situated below a bird cliff, allowing to study the effect of nutrient input by seabirds.

\emph{ITEX site} The ITEX experiment was carried out in the high Arctic, in Endalen (78°11`N, 15°45`E), a valley situated approximately four kilometers east of Longyearbyen, Svalbard, at 80-90 m a.s.l.
Average annual temperature is --5.2°C (1981--2010) and the average annual precipitation is 191 mm (Førland et al., 2011).
The prevailing wind direction is from the east.
The soils are typical Cryosols with thin organic layer on top of inorganic sediments (Jones et al., 2010).
The experiment was established in 2001 in three different habitats located on the south--southeast--facing hillside of the valley.
The habitats differ in vegetation composition and the timing of snowmelt and hence the length of the growing season.
These habitats include a relatively dry Dryas heath with thin snow cover (ca. 10 cm) and early snowmelt, a mesic Cassiope heath habitat with intermediate snow depth and melting date, and a moist snowbed community with deep snow (\textgreater{} 100 cm) and late snowmelt.

This experiment is a part of the International Tundra Experiment (ITEX), a research network established in 1990 to study the long--term responses of tundra plants and vegetation to climate warming (Jónsdóttir, 2004; Molau \& Mølgaard, 1996).
In 2001 30 plots (75x75 cm), were selected, ten in each of the three habitats (Dryas heath, Cassiope heath and snowbed). Half of the plots were randomly assigned to the warming treatment in 2002 and the other half to control. The OTCs have a basal diameter of 1.5 meter and a hight of 40 cm.

\emph{Elevational gradient} Seven transects were established along the elevational gradient, at different elevations along the slope of Lindholmhøgda, near Isdammen. Transects were established in herb-like vegetation, in between snow beds (Cassiope heath) and ridge communities (rocky with very scarce vegetation).

\emph{Birdcliff gradinet} The nutrient deposition gradient was established under a little auk (\emph{Alle alle}, but also a few black guillemots, \emph{Cepphus grylle}) colony near Bjørndalen, up the slope of Platåfjellet (Figure 1, Table 1).
The birds breeding on the cliffs above the site deposit nutrients in the form of guano.
Nutrient deposition will therefore be highest closest to the bird nests as the number of bird droppings per surface area will increase.

In both elevational gradients we established 75 x 75 cm plots arranged in transects (running perpendicular to the slope) containing 7 plots each.
In the bird cliff gradient, we established in total 5 transects (n = 35 plots), in the elevational gradient we established 7 transects, of which the highest transect had only four plots due to limited vegetation available (n = 46 plots).
The plot sizes match the size of permanent plots in the ITEX experiment, so the experiments can be compared.
While selecting the plots, we ensured that plots were placed in areas of similar microtopography / substrate / vegetation.
Plots are mapped with differential GPS and their aspect and slope were detected via digital model of the site.

Plots are localized by GPS coordinates and their aspect and slope were detected via a digital model of the site.

\hypertarget{species-identification-taxonomy-and-flora}{%
\subsubsection{Species identification, taxonomy, and flora}\label{species-identification-taxonomy-and-flora}}

All species sampled in the vegetation and functional trait datasets were identified in the field.
Back at the field station, the identification of sampled whole plants or vouchers were checked by a taxonomic expert from the region using the Flora of Svalbard.
Any problematic species? Dryas???
All taxon names were standardized using the Taxonomic Name Resolution Service38.

\hypertarget{data-collection-methods}{%
\subsubsection{Data collection methods}\label{data-collection-methods}}

\begin{tabular}{l|l|l|l|l|l}
\hline
Responses & Gradients & ITEX & Polyploidy & Fluxtower & Snowfence\\
\hline
Plant community compositon & X & X & NA & NA & NA\\
\hline
Plant functional traits & X & X & X & NA & NA\\
\hline
Bryophyte traits & X & NA & NA & NA & NA\\
\hline
Soil microclimate & X & NA & NA & NA & NA\\
\hline
Leaf physiology & X & ? & NA & NA & NA\\
\hline
CO2 efflux & X & X & NA & NA & NA\\
\hline
Poloploidy & NA & NA & X & NA & NA\\
\hline
Remote sensing and spectral reflectance & X & X & NA & X & X\\
\hline
\end{tabular}

\emph{Plant community composition and vegetation structure}
In the ITEX site, the abundance of all vascular plants, bryophytes and lichens (some bryophytes and lichens only at the genus level) were recorded in each plot using the point intercept method in the growing season of 2003, 2009 and 2015 following the protocols from the ITEX manual (Molau \& Mølgaard, 1996).
A 75 × 75 cm frame with 100 equally distributed points was used and all hits within the canopy were recorded until the pin hit the cryptogam layer (composed of bryophytes and lichens), bare ground, or litter.
Dead plant tissue on the ground was recorded as litter and unvegetated surface was recorded as rock or soil (bare ground).

Along the two elevational gradients the species compositon was recorded in July 2018.
In each plot, the cover of all vascular plant species and bryophytes, biocrust, lichens, litter, rocks and bare ground was visually to the nearest 1\%.
Note that the total coverage in each plot can exceed 100, due to layering of the vegetation.
In addition, we noted if the vascular plants were flowering (fertile?) and measured median vegetation height at five points evenly distributed in each plots using a ruler.

\emph{Plant functional trait collection and measurements}

We measured 12 leaf functional traits that are related to potential physiological growth rates and environmental tolerance of plants, following the standardized protocols in Pérez-Harguindeguy et al.40: plant height (PL, cm), leaf area (LA, cm2), leaf thickness (LT, mm), leaf dry matter content (LDMC, g/g), specific leaf area (SLA, cm2/g), carbon (C, \%), nitrogen (N, \%), phosphorus (P, \%), carbon nitrogen ratio (C:N), nitrogen phosphorus ratio (N:P), carbon13 isotope ratio (δ13C, ‰), and nitrogen15 isotope ratio (δ15N, ‰).

For each plot of the gradients and the ITEX site, we collected three whole individual plants, including roots of each species covering more than 1 \% of each plot for leaf functional trait measurements.
Plant material for trait measurements was collected outside but within the close surroundings of each plot, because we did not want to sample destructively from within the plots.
The individuals were stored water saturated in plastic bags outdoors (temperature around +6°C) to ensure they stay fresh auntil the measurements started.

In the lab, each sample was identified to species level using (xxx literature).
\emph{Plant height} was measured from the bottom of each individual plant to the highest tip of a photosynthetic leaf, excluding florescences but including stem leaves for graminoids.
From each individual at least three leaves were collected.
If the leaves were very small, we collected an amount of leaves equivalent to an area of c. 3 cm\^{}2.
For Equisetum species, which has thin and needle shaped leaves, an 8 cm section was cut off including side shoots and all leaves from this section were removed and used.

Each leaf (with petiole) was weighted to the nearest 0.001 g to assess \emph{fresh mass} using a Mettler AE200, Mettler TOLEDO, and AG204 DeltaRange (0.1 mg precision).

To measure \emph{leaf surface area}, the leaves were carefully patted dry with paper towels, flattened (folded out to their maximum area) and scanned upside down using a Canon LiDE 220 flatbed scanner at 300 dpi.
Overlapping leaves (e.g.~compound leaves) were spread out or cut in parts to obtain the full area of the leaves.
Leaves that could not be flattened because they are narrow and grow naturally folded (e.g., some Festuca sp.) were scanned as they were, thereafter the area was multiplied by 2 during data processing.
The leaf area was calculated using ImageJ (REF) and the R package LeafArea (Katabuchi 2017).

\emph{Leaf thickness} was measured on three different leaves or if fewer leaves were available on the same leaf, using a electronic micrometers (Mitutoyo 293-348).
We avoided measuring thickness on middle veins when possible.

Leaves were then dried for a minimum of 72 hours at 65°C before \emph{dry mass} was measured to the nearest 0.0001 g.

\emph{Leaf stoichiometry and isotopes} A subset of leaves (n = XXX; XXX from the gradient sampling and YYY rom the ITEX experiment) were transported to the University of Arizona for leaf stoichiometry (P, N, C concentration) and isotope assays (δ15N, and δ13C). The leaves were stored in a drying oven at 65 °C before shipping and processing. Each leaf was ground into a fine homogenous powder. Total phosphorus concentration was determined using persulfate oxidation followed by the acid molybdate technique (APHA 1992) and phosphorus concentration was then measured colorimetrically with a spectrophotometer (ThermoScientific Genesys20, USA). Nitrogen, carbon, stable nitrogen (δ15N) and carbon (δ13C) isotopes were measured by the Department of Geosciences Environmental Isotope Laboratory at the University of Arizona on a continuous-flow gas-ratio mass spectrometer (Finnigan Delta PlusXL) coupled to an elemental analyzer (Costech). Samples of 1.0 ± 0.2 mg were combusted in the elemental analyser. Standardization is based on acetanilide for elemental concentration, NBS-22 and USGS-24 for δ13C, and IAEA-N-1 and IAEA-N-2 for δ15N. Precision is at least ± 0.2 for δ15N (1s), based on repeated internal standards. Ratios between C:N and N:P were calculated and used in the analysis.

\begin{itemize}
\tightlist
\item
  Climate data*
\end{itemize}

Loggers from ITEX sites

Tomsts loggers (Pekka, Julia)

\hypertarget{data-records}{%
\subsection{Data Records}\label{data-records}}

\hypertarget{itex}{%
\subsubsection{ITEX}\label{itex}}

\includegraphics{Manuscript_files/figure-latex/itex_diversity-1.pdf}

\end{document}
